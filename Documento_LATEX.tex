\documentclass[12pt,a4paper]{article}

% -------------------------
% Paquetes
% -------------------------
\usepackage[spanish]{babel}
\usepackage[utf8]{inputenc}
\usepackage{amsmath}
\usepackage{graphicx}
\usepackage{hyperref}
\usepackage{geometry}
\usepackage{listings}
\usepackage{xcolor}

\geometry{margin=2.5cm}

\hypersetup{
    colorlinks=true,
    linkcolor=blue,
    urlcolor=blue
}

% Configuración para código Python
\lstset{
    language=Python,
    basicstyle=\ttfamily\small,
    keywordstyle=\color{blue},
    stringstyle=\color{green!60!black},
    commentstyle=\color{gray},
    breaklines=true,
    showstringspaces=false
}

% -------------------------
% Documento
% -------------------------
\title{Funcionamiento Interno y Aplicaciones de GeoPandas}
\author{Dante Márquez López}
\date{\today}

\begin{document}

\maketitle
\tableofcontents
\newpage

% -------------------------------------------------
\section{Introducción}

GeoPandas es una librería de Python diseñada para el análisis y manipulación de datos geoespaciales. 
Extiende la estructura de \texttt{pandas} incorporando una columna geométrica que permite trabajar 
con objetos espaciales como puntos, líneas y polígonos.

Su importancia radica en que integra el análisis tabular con el análisis territorial, 
permitiendo realizar operaciones espaciales avanzadas de manera intuitiva.

% -------------------------------------------------
\section{Estructura Fundamental: GeoDataFrame}

Un \textbf{GeoDataFrame} es una extensión del DataFrame tradicional. 

Mientras un DataFrame clásico contiene columnas numéricas o categóricas, 
un GeoDataFrame añade una columna especial llamada:

\begin{center}
\texttt{geometry}
\end{center}

Esta columna almacena objetos geométricos reales.

\subsection{Tipos de geometrías}

\begin{itemize}
    \item Point
    \item LineString
    \item Polygon
    \item MultiPolygon
\end{itemize}

Ejemplo de creación:

\begin{lstlisting}
import geopandas as gpd
from shapely.geometry import Point

data = {'ciudad': ['A', 'B'],
        'geometry': [Point(-74.1, 4.6), Point(-75.5, 6.2)]}

gdf = gpd.GeoDataFrame(data, crs="EPSG:4326")
\end{lstlisting}

% -------------------------------------------------
\section{Arquitectura Interna}

GeoPandas funciona como un orquestador de múltiples librerías especializadas:

\subsection{Shapely}
Gestiona la representación geométrica y las operaciones espaciales:
\begin{itemize}
    \item Intersección
    \item Unión
    \item Diferencia
    \item Buffer
\end{itemize}

\subsection{PyProj}
Gestiona los sistemas de referencia espacial (CRS) y transformaciones de coordenadas.

\subsection{Fiona}
Permite la lectura y escritura de formatos geoespaciales como:
\begin{itemize}
    \item Shapefile
    \item GeoJSON
    \item GeoPackage
\end{itemize}

% -------------------------------------------------
\section{Sistema de Referencia Espacial (CRS)}

Todo GeoDataFrame tiene un CRS asociado. Este define cómo se interpretan 
las coordenadas en el espacio.

Ejemplo de transformación:

\begin{lstlisting}
gdf = gdf.to_crs(epsg=3857)
\end{lstlisting}

La transformación es fundamental cuando se calculan distancias, áreas o buffers.

% -------------------------------------------------
\section{Operaciones Espaciales}

GeoPandas incorpora operaciones espaciales que no existen en pandas tradicional.

\subsection{Buffer}

\begin{lstlisting}
gdf_buffer = gdf.buffer(1000)
\end{lstlisting}

Genera una zona de influencia alrededor de una geometría.

\subsection{Spatial Join}

\begin{lstlisting}
gdf_resultado = gpd.sjoin(gdf1, gdf2, how="inner", predicate="intersects")
\end{lstlisting}

Permite combinar datasets basándose en relaciones espaciales.

\subsection{Overlay}

\begin{lstlisting}
gdf_diff = gpd.overlay(gdf1, gdf2, how="difference")
\end{lstlisting}

Permite realizar operaciones topológicas entre capas.

% -------------------------------------------------
\section{Modelo Conceptual}

GeoPandas puede entenderse como:

\begin{center}
Datos tabulares + Geometría + Sistema de coordenadas
\end{center}

Matemáticamente, cada fila puede representarse como:

\[
x_i = (atributos_i, geometría_i)
\]

Donde la geometría es un objeto definido en un espacio métrico.

% -------------------------------------------------
\section{Aplicaciones en Ciencia de Datos}

GeoPandas conecta el análisis estadístico con el análisis territorial:

\begin{itemize}
    \item Planeación urbana
    \item Optimización logística
    \item Análisis de cobertura
    \item Marketing territorial
    \item Generación de microzonas
\end{itemize}

Permite transformar datos crudos en información espacialmente contextualizada.

% -------------------------------------------------
\section{Conclusión}

GeoPandas no es simplemente una herramienta para visualizar mapas. 
Es una extensión del paradigma de análisis de datos hacia el espacio geográfico.

Permite integrar estadística, geometría y análisis territorial 
dentro del ecosistema de Python.

\vspace{1cm}

Repositorio de ejemplo:
\url{https://github.com/MLDante/Ejemplo-GeoPandas.git}

\end{document}